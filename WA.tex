\documentclass[11pt]{scrreprt}

\usepackage{graphicx}
\usepackage[utf8]{inputenc}
\usepackage[onehalfspacing]{setspace}
\usepackage[ngerman]{babel} 
\usepackage[utf8]{inputenc}
\usepackage[babel, german=quotes]{csquotes}
\usepackage{amsmath}

\title{Wie Kartendienste ans Ziel finden}
\date{06. Juli 2020}
\author{Johannes Quast}

\begin{document}
\begin{titlepage}
	\centering
	\includegraphics[width=0.33\textwidth]{Bilder/Arrow}\par\vspace{1cm}
	{\scshape\LARGE DHBW Karlsruhe \par}
	\vspace{1cm}
	{\scshape\Large Wissenschaftliches Arbeiten\par}
	\vspace{1.5cm}
	{\huge\bfseries Wie Kartendienste zum Ziel finden\par}
	\vspace{2cm}
	{\Large\itshape Johannes Quast\par}
	\vfill
	supervised by\par
	Dr.~Mark \textsc{Brown}
	
	\vfill
	
	% Bottom of the page
	{\large \today\par}
\end{titlepage}

\tableofcontents

\chapter{Einleitung}
	\section{Motivation}
	Ich glaube jeder von uns kennt die folgende Situation: \\
	Man möchte mit dem Auto oder Verkehrsmittel seiner Wahl an einen bestimmten Ort gelangen, an dem man vorher noch nie gewesen ist.
	Die Lösung dieses Problems heißt heutzutage in den meisten Fällen \enquote{Google Maps} oder trägt den Namen einer anderen populären Kartenapp.
	Kaum hat meine eine beliebe Adresse in ein Suchfeld eingegeben, bekommt man bereits nach wenigen Sekunden eine Route vorgeschlagen.
	\\\\
	In Zeiten der Digitalisierung, in der solche Apps für uns selbstverständlich geworden sind und klassische Navigationsmethoden wie z.B. Atlanten oder Karten des Straßennetzes beinahe ausgestorben sind, fragt sich kaum jemand, wie solche Routenplaner eigentlich funktionieren.
	In dieser Ausarbeitung möchte ich genau auf diese Hintergründe eingehen und versuchen zu erklären, wie eine solche Routenberechnung eigentlich funktioniert und welche Möglichkeiten es gibt, diesen grundlegenden Algorithmus zu beschleunigen.


	\section{Vorwort}
	Bevor man sich mit der Routenplanung in einem Straßennetz im speziellen beschäftigt, sollte man dieses Problem auf ein etwas allgemeineres reduzieren: Die Wegfindung von einem Punkt X zu einem Punkt Y.
	Es sollte natürlich selbsterklärend sein, dass Punkt X und Y meistens keine direkte Verbindung zueinander haben, sondern die Route über zahllose Zwischenpunkte gehen wird.
	\\\\
	Um dieses Problem zu lösen gibt es sehr viele Ansätze, aber nicht alle sind praktikabel aufgrund verschiedener Limitationen der jeweiligen Ansätze.
	Auf den folgenden Seiten werden wir einige Möglichkeiten betrachten, sodass am Ende klar sein sollte, welche Möglichkeiten für eine Routenplanung in einem Straßennetz in Frage kommen.

\chapter{Grundlagen}
	\section{Graphen}
	Fast genauso wichtig wie das Verständnis für den Algorithmus ist ein grober Überblick darüber, auf welchen Daten bzw. Datenstrukturen unser Algorithmus arbeiten wird.
	Die Datenstruktur, die für diesen Anwendungszweck in Frage kommt nennt sich \textit{Graph}.
	Vereinfacht gesagt besteht ein Graph aus Knoten (z.B. Punkt X oder Y) und Kanten (Verbindungen zwischen den Knoten).

	\begin{figure}[ht]
		\centering
		\includegraphics[width=0.7\textwidth]{Graphs/hello}
		\caption{Ein einfacher Graph mit Knoten (A,B,C,...) und Kanten}
	\end{figure}

	Wie man in der Abbildung sehen kann, sind alle Kanten zusätzlich mit Zahlen versehen. Diese geben meistens eine Art \textit{Gewichtung} o.ä. an. Im Kontext der Wegfindung können diese Zahlen zum Beispiel den Aufwand angeben, den jemand benötigt um z.B. von Punkt A zu B zu gelangen. In der realen Welt kann dieser Aufwand beispielsweise mit der Länge der Strecke oder Fahrzeit verbunden werden.
	\newpage

	\section{Prioritätenliste (priority queue)}
	Diese Datenstruktur ist wichtig für viele Algorithmen, die den kürzesten Weg finden möchten.
	Wie der Name bereits vermuten lässt, handelt es sich hierbei um eine Warteschlange, die bestimmte Objekte speichern kann.
	Diese werden nach einer bestimmten \textit{Priorität} sortiert.

	\begin{figure}[ht]
		\centering
		\includegraphics[width=0.8\textwidth]{Graphs/pq_objects}
		\caption{Objekte mit Fahrzeit und Entfernung zwischen zwei Städten}
	\end{figure}
	
	Würde man diese 3 Objekte nun nacheinander in eine Prioritätswarteschlange einfügen, dann würden diese sich automatisch an die richtige Position in der Warteschlange einreihen.
	
	\begin{figure}[ht]
		\centering
		\includegraphics[width=0.8\textwidth]{Graphs/pq_sorted_distance}
		\caption{Objekte mit Fahrzeit und Entfernung zwischen zwei Städten}
	\end{figure}

\chapter{Algorithmen}
Auf den folgenden Seiten möchte ich mich auf zwei sehr bekannte Algorithmen beschränken, da diese ein sehr gutes Verständnis dafür liefern, wie eine Routenberechnung implementiert werden kann und wie die Wegfindung vorgeht.

	\section{Dijkstra}
	Der erste Algorithmus ist der Dijkstra Algorithmus, auf dem auch der A*-Algorithmus basiert.
	Er arbeitet auf einem Graphen um die kürzeste Route zwischen einem gegebenen Startknoten und einem bestimmten Zielknoten zu ermitteln. Jede Kante im Graphen muss dabei eine nicht negative Gewichtung aufweisen.
	

		\subsection{Allgemeine Funktionsweise}
		\begin{enumerate}
			\item Bevor die eigentliche Wegfindung beginnt, wird die Datenstruktur initialisiert.	Dabei wird jedem Knoten in unserem Graphen folgende zwei Eigenschaften zugeordnet: Entfernung und Vorgänger. Die Entfernung gibt dabei die Distanz zum Startknoten an. Es ist an dieser Stelle zu erwähnen, dass der Begriff \enquote{Entfernung} hier stark von der verwendeten Kantengewichtung abhängt, denn nicht immer ist die Einheit dieser Werte ein Weg o.ä.
			Alle Knoten haben eine initiale Entfernung von $\infty$, da am Anfang der Suche noch nicht sichergestellt ist, ob ein bestimmter Knoten vom Startknoten überhaupt erreichbar ist.
			Der Startknoten selbst bekommt die Entfernung 0 zugeordnet.
			
			\item In diesem und den folgenden Schritten kommt die zuvor erklärte \textit{Prioritätenliste} zum Einsatz. Alle Knoten werden in dieser Liste gespeichert und aufsteigend nach der Entfernung sortiert.
			
			\begin{enumerate} 
				\item Nun beginnt man, für alle Nachbarknoten des aktiven Knotens (am Anfang ist dies der Startknoten) die Entfernungen zu berechnen. 
				\begin{equation*}
					Enterfnung_{Knoten_{Nachbar}} = Enterfnung_{Knoten_{Aktiv}} + Kantengewicht
				\end{equation*}
				Diese berechnete Entfernung wird nur gesetzt, wenn die bereits gesetzte Entfernung des Nachbarknotens größer ist, als die neue. Sollte dies der Fall sein, wird dieser Wert aktualisiert und das Attribut \textit{Vorgänger} wird auf den aktiven Knoten gesetzt.
				\item Der momentan aktive Knoten wird nun der Liste der \textit{besuchten Knoten} hinzugefügt. Dies hat den einfachen Hintergrund, dass der Algorithmus nicht mehrmals den selben Knoten untersucht, oder ein Zirkel entsteht (z.B. wenn der Algorithmus auf einen Kreisverkehr trifft).
				\item Jetzt beginnt die Schleife erneut bei Schritt a. Allerdings mit dem wichtigen Unterschied, dass der aktive Knoten mit dem ersten Knoten aus der \textit{Prioritätenliste} ersetzt wird. Kurz zur Erinnerung: Der 1. Knoten dieser Warteschlange ist der Knoten, der momentan die geringste Entfernung zum Startknoten ausweist. Die Schleife \textbf{terminiert}, wenn der aktive Knoten gleich dem gesuchten Zielknoten entspricht.
			\end{enumerate}
			
		\end{enumerate}
	
		\newpage
		\subsection{Beispiel}
		Für das Beispiel verwenden wir den 1. Beispielgraphen. Der Startknoten in diesem Fall ist \textit{A}, der Zielknoten \textit{G}. Der momentan aktive Knoten wird blau markiert, die bereits besuchten Lila. Orange sind die Nachbarknoten, für die gerade die Entfernung berechnet wurde (Vergleich Schritt c. der allgemeinen Funktion).
		Weiterhin werden die Einträge der Prioritätenliste visualisiert, indem jeweils der Name des Knoten selbst, die Entfernung zum Startknoten und über welchen Vorgängerknoten man den Startknoten erreichen kann, dargestellt wird.
			\subsubsection{1. Schritt}
			\begin{figure}[ht]
				\centering
				\includegraphics[width=0.8\textwidth]{Graphs/Example1}
				\caption{Objekte mit Fahrzeit und Entfernung zwischen zwei Städten}
			\end{figure}
			\begin{figure}[ht]
				\centering
				\includegraphics[width=0.3\textwidth]{Graphs/Example1_PQ}
				\caption{Prioritätenliste nach Schritt 1}
			\end{figure}
			Der Knoten A ist in diesem Fall der Startknoten und hat die Nachbarknoten B und C.
			Für beide Nachbarn wird nun die Entfernung berechnet, ausgehend von A.
			\begin{align*}
					Enterfnung_B &= Enterfnung_A + 3 = 0 + 3 = 3\\
					Entfernung_C &= Entfernung_A + 5 = 0 + 5 = 5
			\end{align*}
			
			\subsubsection{2. Schritt}
			\begin{figure}[ht]
				\centering
				\includegraphics[width=0.8\textwidth]{Graphs/Example2}
				\caption{Objekte mit Fahrzeit und Entfernung zwischen zwei Städten}
			\end{figure}
			\begin{figure}[ht]
				\centering
				\includegraphics[width=0.45\textwidth]{Graphs/Example2_PQ}
				\caption{Prioritätenliste nach Schritt 2}
			\end{figure}
			Der Knoten A ist in diesem Fall der Startknoten und hat die Nachbarknoten B und C.
			Für beide Nachbarn wird nun die Entfernung berechnet, ausgehend von A.
			\begin{align*}
			Enterfnung_B &= Enterfnung_A + 3 = 0 + 3 = 3\\
			Entfernung_C &= Entfernung_A + 5 = 0 + 5 = 5
			\end{align*}
			

\section{A*}

\subsection{Funktionsweise}

\chapter{Mögliche Verbesserungen}

\end{document}
